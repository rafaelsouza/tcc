\chapter{Ferramentas Avaliadas}

As ferramentas avaliadas pertencem ao grupo dos Web Content Management Systems, são escritas em Ruby on Rails, e foram as mais populares encontradas no site github.com .

\section{RadiantCMS}

\begin{quote}
Radiant é um sistema de gerenciamento de conteúdo aberto, simples e projetado para pequenas equipes
\cite{radiant_website}\footnote{Tradução livre do autor}
\end{quote}

Radiant é um dos CMS mais antigos disponíveis para Ruby, lançado em 2006. De instalação e uso simples tem uma grande quantidade de plugins disponíveis e uma grande comunidade mantenedora da aplicação.

\subsection{Principais Características do Radiant}

\begin{itemize}
  \item Interface intuitiva
  \item Bom sistema de extensões, com um número muito grande de extensões disponíveis.
  \item Sistema de usuários e permissões simples.
  \item Snippets
  \item Linguagem específica de domínio Radius.
\end{itemize}

A Figura \ref{fig:radiant_admin} mostra a interface de administração do Radiant, que é relativamente simples.

\begin{figure}[here]
\includegraphics[width=150mm]{images/radiant_admin.png}
\caption{Interface de administração do Radiant}
\label{fig:radiant_admin.png}
\end{figure}

Radiant tem um sistema de usuários e permissões simplificado, bom para pequenas organizações que não necessitam de grande complexidade no acesso. Possui algumas extensões de código aberto que adicionam um maior leque de permissões e funcionalidades ao sistema.

Os Snippets são pedaços reutilizáveis de código, que podem ser inseridos em qualquer página ou template (layout), são úteis e dentro da filosofia de Rails DRY, Don't Repeat Yourself. 

Radius é uma linguagem específica de domínio escrita em Ruby, baseada em tags, semelhante a XML e HTML. É utilizada dentro do Radiant para prover algumas funcionalidades como os snippets e outras marcações que encapsulam parte da lógica da aplicação. É possível a partir dela gerar qualquer forma de texto puro ou html. 

\section{RefineryCMS}

RefineryCMS é um sistema gerenciador de conteúdos desenvolvido pela Resolve Digital. Ele é um sistema que começou em 2005 e que em meados de 2009 abriu o seu código para a comunidade. Projetado para pequenos projetos onde pessoas possam manter o conteúdo sem muita complexidade. Possui uma interface amigável, simples de se trabalhar, com um editor de conteúdo WYISWYM (What You See Is What You Mean) integrado. Sua instalação é simples e a documentação é pequena porém bem organizada. Ele recentemente foi portado para  Ruby on Rails versão 3. Ele utiliza engines, que são mini aplicações Rails dentro de um projeto Rails, como forma de extensão, facilitando para a comunidade escrever extensões para a ferramenta. 

O sistema de modelos (templates) do Refinery é feito através de temas, gerados a partir de seu modelo de extensões. Seu ponto forte é que não é necessário aprender uma linguagem específica de domínio como o Radius para efetuar as customizações de layout. Uma desvantagem é que o usuário final pouco pode customizar um layout dentro da interface de administração.

\subsection{Principais Características do Refinery}

\begin{itemize}
  \item Interface intuitiva, fácil para usuários não técnicos.
  \item Sistema de extensões baseado em engines, fácil de ser usado, porém não há muitas engines disponíveis.
  \item Instalação simples
  \item Documentação bem organizada.
  \item O foco do usuário final é somente gerenciar conteúdo.
\end{itemize}

A Figura \ref{fig:refinery_admin} mostra o painel de controle do Refinery, que mostra o acesso rápido as ações mais comuns ao gerenciar conteúdo. A Figura \ref{refinery_new_page} mostra a interface de gerenciamento de um conteúdo do Refinery.

\begin{figure}[here]
\includegraphics[width=150mm]{images/refinery_admin.png}
\caption{Interface de administração do Refinery}
\label{fig:refinery_admin}
\end{figure}

\begin{figure}[here]
\includegraphics[width=150mm]{images/refinery_new_page.jpg}
\caption{Interface de criação / edição de página do Refinery}
\label{fig:refinery_new_page}
\end{figure}

\section{BrowserCMS}

BrowserCMS é um sistema gerenciador de conteúdos desenvolvido pela empresa BrowserMedia, encontra-se na versão 3.1.2. Foi, por vários anos um CMS comercialmente licensiado, escrito na linguagem Java. Motivados pela "onda" de Ruby on Rails, e pela falta de um CMS em Ruby on Rails que suprisse as necessidades de seus clientes, reescreveram o seu CMS em 2009 e deixaram-o com uma licensa de código aberto.

\subsection{Principais Características do BrowserCMS}

\begin{itemize}
  \item Sistema voltado ao mundo corporativo para grandes e médias organizações.
  \item Sistema de workflow completo de publicação.
  \item Sistema amplo de permissões.
  \item Versionamento de conteúdo.
  \item Poucas extensões 
  \item Documentação pequena.
\end{itemize}

A figura \ref{browser_cms_new_page} mostra a criação de uma página no Browsercms, notem que ela é desvinculada do conteúdo, diferentemente do que acontece com Radiant e Refinery, devido a possibilidade de reuso do conteúdo. A Figura \ref{browsercms_new_content} mostra a interface de criação de conteúdo do BrowserCMS.

\begin{figure}[here]
\includegraphics[width=150mm]{images/browser_cms_new_page.jpg}
\caption{Interface de criação de página do BrowserCMS}
\label{fig:browsercms_new_page}
\end{figure}

\begin{figure}[here]
\includegraphics[width=150mm]{images/browser_cms_new_content.jpg}
\caption{Interface de criação de conteúdo do BrowserCMS}
\label{fig:browsercms_new_content}
\end{figure}


