\chapter{Ruby}

\begin{quote} 
Ruby é uma linguagem de programação de código aberto e orientada a objetos criado por Yukihiro 'Matz' Matsumoto. A primeira versão foi lançada no Japão em 1995. Ruby tem ganhado aceitação ao redor do mundo como uma linguagem fácil de aprender, poderosa, e expressiva, especialmente desde o advento do Ruby on Rails, um framework para aplicações voltadas para Web escrito em Ruby\footnote{http://www.rubyonrails.org}. O núcleo de Ruby é escrito na linguagem de programação C e roda na maioria das plataformas. É uma linguagem interpretada e não compilada.
\cite{ruby_pocket_reference}
\end{quote}
\footnote{Tradução livre do autor}

Uma descrição pouco mais detalhada sobre Ruby: 
\begin{quote}
Ruby é uma linguagem de programação dinâmica com uma complexa, porém expressiva gramática e um núcleo de  biblioteca de classes com uma rica e poderosa API. Ruby tem inspirações em Lisp,SmallTalk e Perl, mas usa uma gramática que facilita o entendimento de programadores Java e C. Ruby é uma linguagem orientada à objetos pura, mas também suporta estilos de programação funcional e procedural.Ela inclui uma poderosa capacidade de metaprogramação e pode ser usada para criar linguagens de domínio específico ou DSLs.} 
\cite{the_ruby_proggraming_language}
\end{quote}
\footnote{Tradução livre do autor}

Ruby é escrita em C e atualmente está disponível para as mais diversas plataformas UNIX, Mac OS X, Windows 95/98/Me/NT/2000/XP/Vista/Seven, DOS, BeOS, OS/2, .NET, Solaris (\cite{ruby_official_website}).

O criador de Ruby, Yukihiro 'Matz' Matsumoto, teve a idéia de criar uma linguagem de programação em 1993, durante uma conversa um um colega sobre linguagens de script. Através da conversa, o seu interesse cresceu muito sobre as linguagens de script, onde ele ficou impressionado pelo poder e pelas possibilidades. Como fora um fã de longa data de programação orientada à objetos, lhe pareceu , a orientação à objetos, muito adequada para linguagens de script. Então passou olhar pela rede e encontrou Perl 5, que ainda não havia sido lançada, contendo algumas características de orientação à objetos, mas ainda não era o que ele queria. Então decidiu abandonar Perl como uma linguagem de script orientada à objetos. Então ele procurou Python, era uma linguagem orientada à objetos interpretada. Mas ele não teve a sensação que era uma linguagem de script, era uma linguagem híbrida, procedural e orientada à objetos. Matz queria uma linguagem de script que seria mais poderosa que Perl e mais orientada à objetos que Python. Então ele decidiu criar a própria linguagem. O nome Ruby veio de uma brincadeira com um amigo durante o projeto de desenvolvimento da linguagem, ele queria um nome de uma pedra preciosa, a là Perl, e então o amigo sugeriu Ruby, que depois se tornou o nome oficial da linguagem.  (\cite{inverview_with_matz}).

A filosofia de Ruby segundo o Yukihiro 'Matz' Matsumoto: \emph{Ruby foi projetado para fazer programadores mais felizes}
\cite{the_ruby_programming_language}{página 1}\footnote{Tradução livre do autor}

Abaixo um exemplo de uma classe Ruby 

No exemplo que segue podemos visualizar como Ruby define uma classe e cria um objeto. A definição de classe se dá pelo comando \emph{class}. No exemplo o nome da classe é \emph{Pessoa}. Na linha 2 há uma declaração de um atributo da classe \emph{attr_acessor}, esta declaração faz com que o Ruby gere automaticamente métodos de acessores e modificadores.  O método de criação do objeto chamado \emph{initialize} recebe um argumento \emph{nome}. O exemplo ilustra uma das características de Ruby que é a possibilidade de valor padrão para um argumento. Ao criar o objeto com o método \emph{new} pode se passar um objeto string com um nome, ou então irá ser utilizado o padrão. Dentro do método initialize iremos apontar o argumento sexo para a variável de classe \emph{@sexo}.

Nas linhas 15 e 16 criaremos 2 objetos da classe pessoa passando argumentos diferentes para cada um deles. Nas linhas subsequentes iremos escrever na tela do console o valores retornados pelo método agradecer que vai retornar o agradecimento correto para \emph{homem} e \emph{mulher}.

\lstinputlisting[
  language=Ruby,
  emph={catch,class,def,else,end,if,require,return,then,throw,unless,attr_acessor},
  emphstyle=\color{Red}\bfseries,
  emph={[2]initialize,new},
  emphstyle={[2]\color{Blue}},
  emph={[3]Pessoa},
  emphstyle={[3]\color{Yellow}\bfseries},
  emph={[4]@nome},
  emphstyle={[4]\color{Green}\bfseries},
  basicstyle=\small,
  title=\lstname,
  numbers=left
]{code/exemplo_ruby.rb}


Em Ruby, como é uma linguagem orientada à objeto pura, tudo é um objeto. Desde números aos valores booleanos true  e false ao nil (versão de Ruby para null, indicação de falta de um valor).(\cite{the_ruby_programming_language)

O exemplo abaixo demonstra que valores numéricos, valores booleanos, e o nil podem invocar o método \emph{class} que retorna a sua classe e imprimiremos o resultado na tela do console.

\lstinputlisting[
  language=Ruby,
  emph={catch,class,def,else,end,if,require,return,then,throw,unless},
  emphstyle=\color{Red}\bfseries,
  emph={[2]initialize,new},
  emphstyle={[2]\color{Blue}},
  emph={[3]Pessoa},
  emphstyle={[3]\color{Yellow}\bfseries},
  emph={[4]@nome},
  emphstyle={[4]\color{Green}\bfseries},
  basicstyle=\small,
  title=\lstname,
  numbers=left
]{code/exemplo_objetos.rb}

O resultado impresso na tela pelo exemplo é:\\
\$ ruby exemplo_objetos.rb\\
Fixnum\\
TrueClass\\
NilClass\\

Cada linha é a resposta do nome da classe ao qual cada valor pertence. Por exemplo o número 1 é da classe \emph{Fixnum}, o número 1.1 pertence a classe \emph{Float} e assim por diante.

O interpretador mais usado e conhecido de Ruby é o MRI, ou Matz Ruby Interpreter, escrito em C e a sua versão atual é 1.9.2. 
Além dele há outros interpretadores como JRuby, Rubinius, Maglev, MacRuby e IronRuby. 
\subsection{JRuby}
JRuby é um interpretador Ruby rodando sobre a máquina virtual Java (JVM). Ele está na versão 1.5.2 que é compatível com o MRI 1.8.7.

\subsection{Rubinius}
Rubinius que foi originalmente idealizado para ser o interpretador de Ruby feito em Ruby, mas em virtude da compatibilidade com o MRI parte do seu código é escrito em C++. Está na versão 1.0.1 e é compatível ao MRI 1.8.7 \emph{Um grande aspecto de linguages populares como Java e C, é que a maioria das funcionalidades disponíveis para os programadores é escrito na própria linguagem. O objetivo de Ribinius é adicionar Ruby à estas linguagens. Desta forma Rubystas podem facilmente adicionar funcionalidades à linguagem, arrumar defeitos, e aprender como a linguagem funciona. Onde é possível Rubinius é escrito em Ruby. Onde (ainda) não é escrito em C++.}
\cite{rubinius_official_site}\footnote{Tradução livre do autor}

\subsection{Maglev}
Maglev é o um interpretador Ruby desenvolvido pela empresa Gemstone. Conhecida pelo seu interpretador de Smalltalk, a empresa se lançou como uma plataforma alternativa, já que inclui um mecanismo de persistência de objetos distribuídos. Está em versão Alpha ainda.\emph{Maglev é uma implementação de Ruby rápida e estável com uma integrada persistência de objetos e cache compartilhado e distribuído.}
\cite{maglev_official_site}\footnote{Tradução livre do autor}

\subsection{MacRuby}
Falar sobre MacRuby

\subsection{IronRuby}
IronRuby que é um interpretador Ruby feito sobre a plataforma .NET. Ele era patrocinado pela Microsoft até Julho de 2010, onde a empresa cortou o último desenvolvedor remanescente do projeto. Está na versão 1.0 e com um futuro incerto pois ainda não apareceu nenhum outro mantenedor oficial.

