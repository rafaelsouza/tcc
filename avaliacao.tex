\chapter{Avaliação das ferramentas}

Este capítulo se dedica a fazer a avaliação entre as ferramentas Radiant, Refinery, BrowserCMS, conforme os aspectos apresentados anteriormente.

\section{Qualification and Selection of Open Source Sofware}

O critério observado da metodologia QSOS foi o de gerenciamento de conteúdo na web (Web Content Management), sua origem se encontra em \url{http://www.qsos.org/templates/cms.html}. A avaliação foi executada pelo autor do trabalho. 

\begin{table}[ht]
\caption{Tabela de avaliação do critério de gerenciamento de conteúdo da QSOS } % title of Table
\centering % used for centering table
\begin{tabular}{p{4.2cm} | c | c | c} % centered columns (5 columns)
\hline\hline %inserts double horizontal lines
Critério & Radiant & Refinery & BrowserCMS  \\ [0.5ex] % inserts table
% heading
\hline % inserts single horizontal line
Blog        & 1  & 1   & 1    \\ \hline
Recent changes advertised on the web interface  & 0 & 1 & 0    \\ \hline
Support for other notifications media & 0 & 0 & 0 \\ [1ex] % [1ex] adds vertical space
\hline % inserts single line
\end{tabular}
\label{table:qsos_avaliacao_ferramentas} % is used to refer this table in the text
\end{table}

A tabela\ref{table:qsos_avaliacao_ferramentas} mostra que no quesito blog houve um score parcial em todas as ferramentas, já que todos trazem esta funcionalidade como extensão. Há um resultado positivo para o Refinery no quesito de atualizações recentes pois ele é o único que possui um dasboard informando as últimas atividades. E no terceiro item todas as ferramentas falharam, na notificação de atualizações por outros formatos como email e rss.

\section{Avaliação da Comunidade Open Source}

Esta tabela de avaliação se baseia em dados retirados dos repositórios de dados (\url{http://www.github.com}), lista de discussão de email, e redes sociais como o Twitter (\url{http://www.twitter.com})

\begin{table}[ht]
\caption{Tabela comparativa da comunidade open source de cada ferramenta } % title of Table
\centering % used for centering table
\begin{tabular}{c c c c} % centered columns (5 columns)
\hline\hline %inserts double horizontal lines
Critério & Radiant & Refinery & BrowserCMS \\ [0.5ex] % inserts table
% heading
\hline % inserts single horizontal line
Watchers               & 1087  & 897   & 711   \\ % inserting body of the table
Forks                  & 215   & 219   & 84    \\
Commits                & 399   & 3090  & 99    \\
Contribuidores         & 52    & 58    & 17    \\ 
Pageviews              & 81942 & 70588 & 8259 \\ 
Lista de email         & Sim   & Sim   & Sim  \\ 
Número pessoas email   & 409   & 344   & 369  \\ 
Número mensagens lista & 1716  & 1653  & 859  \\ 
Twitter                & Sim   & Sim   & Sim  \\ 
Seguidores twitter     & 276   & 45    & 308  \\ 
IRC                    & Sim   & Sim   & Sim  \\ [1ex] % [1ex] adds vertical space
\hline % inserts single line
\end{tabular}
\label{table:nonlin} % is used to refer this table in the text
\end{table}

Pode-se observar nesta tabela que o Radiant e Refinery se destacam quanto ao número de pessoas que acompanham o projeto, fazem suas versões do mesmo, contribuem e participam da lista de email. São de fato os dois projetos mais ativos nestes termos. BrowserCMS fica um pouco aquém dos dois . É visível que todas as ferramentas possuem variados canais de comunicação com seus usuários (todos os canais avaliados), ressalta a importância dada à comunidade pelos seus mantenedores.

\section{Avaliação por checklist}

A avaliação pela checklist apresentada no capítulo anterior trouxe resultados interessantes. Pelo fato dela ser muito longa e em formato excel, as checklists de cada ferramenta estão disponíveis nesta url \url{https://github.com/rafaelsouza/tcc/tree/master/checklists/}. 

As imagens com a tabela e o gráfico do resultado da checklist contém dois critérios não utilizados na checklist (Trust \& Credibility, Writing \& Content Quality), pois está fora do escopo desta avaliação.

Abaixo há o resultado da avaliação de cada ferramenta e os pontos importantes levantados pela inspeção por checklist efetuada. 

A inspeção das três ferramentas foi executada pelo autor.

\subsection{Radiant}

\begin{figure}[here]
\includegraphics[width=130mm]{images/radiant_result_checklist_table.jpg}
\caption{Tabela com o resultado da inspeção por checklist no Radiant}
\label{fig:resultado_checklist_radiant_tabela}
\end{figure}

Esta tabela foi gerada a partir da checklist e mostra um resultado geral, o qual ressalta a ausência do campo de busca nas funções da ferramenta. Nota-se também um desempenho abaixo da média no critério de Task Orientation.

Porém, salienta-se o bom desempenho nos critérios de Page Layout \& Visual Design e Home Page.

A Figura \ref{fig:resultado_checklist_radiant_grafico} demonstra o resultado em forma de gráfico para uma melhor visualização. Quanto maior a área pintada maior o score obtido nos quesitos. Cada área pintada compreende um conjunto de dois critérios. 

\begin{figure}[here]
\includegraphics[scale=0.5]{images/radiant_result_checklist_graph.jpg}
\caption{Gráfico com o resultado da inspeção por checklist no Radiant}
\label{fig:resultado_checklist_radiant_grafico}
\end{figure}

Alguns pontos que se destacaram durante a inspeção foram as mensagens de confirmação como mostra a Figura \ref{fig:confirmacao_radiant},  e de erro como demonstrada na Figura \ref{fig:erro_radiant}. Outro ponto de destaque foram as suas opções dentro da tela de criação / edição de conteúdo para salvar, salvar e continuar editando ou cancelar, mostrado na Figura \ref{fig:salvar_radiant}.

Pontos que se destacaram pelo baixo aproveitamento de pontuação, foram principalmente a falta da funcionalidade de busca, mostrado na Figura \ref{fig:falta_busca_radiant} e a ausência de opções para editar o conteúdo de forma mais robusta e usual, pois o sistema suporta apenas edição direta de html ou filtros como MarkDown e Textile, demonstrado na Figura \ref{fig:falta_editor_radiant} 

\begin{figure}[here]
\includegraphics[width=100mm]{images/radiant_ponto_opcoes.jpg}
\caption{Opções para salvar conteúdo no Radiant }
\label{fig:salvar_radiant}
\end{figure}

\begin{figure}[here]
\includegraphics[width=100mm]{images/radiant_ponto_mensagem.jpg}
\caption{Mensagem de confirmação de ação do Radiant }
\label{fig:confirmacao_radiant}
\end{figure}

\begin{figure}[here]
\includegraphics[width=170mm,height=90mm]{images/radiant_ponto_msg_erro.jpg}
\caption{Mensagem de erro do Radiant }
\label{fig:erro_radiant}
\end{figure}


\begin{figure}[here]
\includegraphics[width=150mm]{images/radiant_erro_busca.jpg}
\caption{Falta de busca no radiant}
\label{fig:falta_busca_radiant}
\end{figure}

\begin{figure}[here]
\includegraphics[width=150mm]{images/radiant_erro_conhecimento.jpg}
\caption{Ausência de um editor de html mais robusto para a criação de conteúdo}
\label{fig:falta_editor_radiant}
\end{figure}



\subsection{Refinery}

\begin{figure}[here]
\includegraphics[width=130mm]{images/refinery_result_checklist_table.jpg}
\caption{Tabela com o resultado da inspeção por checklist no Refinery}
\label{fig:resultado_checklist_refinery_tabela}
\end{figure}

A Figura \ref{fig:resultado_checklist_refinery_tabela} contendo a tabela gerada a partir da checklist, destaca-se o alto índice nos critérios da Home Page, Page Layout \& Visual Design e Task Orientation. Os destaques com score abaixo da média ficaram nos critérios de Search e de Help,Feedback \& Error Tolerance. 

A Figura \ref{fig:resultado_checklist_refinery_grafico} demonstra o resultado em forma de gráfico para uma melhor visualização.

\begin{figure}[here]
\includegraphics[scale=0.5]{images/refinery_result_checklist_graph.jpg}
\caption{Gráfico com o resultado da inspeção por checklist no Refinery}
\label{fig:resultado_checklist_refinery_grafico}
\end{figure}

Os pontos que se destacaram durante a inspeção da ferramenta foram a presença dos atalhos na página principal do Refinery, demonstrado na Figura \ref{fig:dashboard_refinery}. As ajudas (tooltips) presentes em vários labels do mesmo, como mostra a Figura \ref{fig:tooltip_refinery} foram outro ponto de destaque.

Pontos que se destacaram com score abaixo da média foram a funcionalidade de busca, embora presente, é uma busca simples sem maiores detalhes. A Figura \ref{fig:cabecalho_refinery} mostra a ausência de uma indicação clara de onde o usuário se encontra, se está editando ou criando um conteúdo. A Figura \ref{fig:campos_obrigatorios_refinery} ilustra a falta de distinção entre campos obrigatórios e não obrigatórios. No caso o campo title é obrigatório e não vem com indicação de sua obrigatoriedade.

\begin{figure}[here]
\includegraphics[width=150mm]{images/refinery_ponto_dashboard.jpg}
\caption{Painel com acesso rápido as funções do Refinery}
\label{fig:dashboard_refinery}
\end{figure}

\begin{figure}[here]
\includegraphics[width=150mm]{images/refinery_ponto_tooltip.jpg}
\caption{Tooltip de ajuda no Refinery}
\label{fig:tooltip_refinery}
\end{figure}


\begin{figure}[here]
\includegraphics[width=150mm]{images/refinery_erro_cabecalho.jpg}
\caption{Falta indicação clara de onde o usuário se encontra no Refinery }
\label{fig:cabecalho_refinery}
\end{figure}

\begin{figure}[here]
\includegraphics[width=150mm]{images/refinery_erro_campos_obrigatorios.jpg}
\caption{Falta uma indicação de campo obrigatório no Refinery. }
\label{fig:campos_obrigatorios_refinery}
\end{figure}


\subsection{BrowserCMS}

\begin{figure}[here]
\includegraphics[width=130mm]{images/browsercms_result_checklist_table.jpg}
\caption{Tabela com o resultado da inspeção por checklist no BrowserCMS}
\label{fig:resultado_checklist_browsercms_tabela}
\end{figure}

Esta Figura \ref{fig:resultado_checklist_browsercms_tabela} contendo a tabela gerada a partir da checklist, destaca-se o alto índice no critérios de Navigation & IA. 

Os destaques negativos ficaram nos critérios da Search,Forms \& Data Entry,Help Feedback \& Error Tolerance. 

A Figura \ref{fig:resultado_checklist_browsercms_grafico} demonstra o resultado em forma de gráfico para uma melhor visualização.

\begin{figure}[here]
\includegraphics[scale=0.5]{images/browsercms_result_checklist_graph.jpg}
\caption{Gráfico com o resultado da inspeção por checklist no BrowserCMS}
\label{fig:resultado_checklist_browsercms_grafico}
\end{figure}

O BrowserCMS traz uma proposta de interface diferente de Radiant e Refinery, onde o usuário edita o conteúdo fazendo diretamente uma pré visualização do mesmo. Há uma curva de aprendizado para a ferramenta, mas após acostumado ela traz alguns benefícios como a possibilidade de reutilizar conteúdos de outras páginas, facilitando a manutenção do conteúdo.

Um ponto a se destacar, está na Figura \ref{fig:browsercms_neutro_icones} onde ilustra a possibilidade de editar o conteúdo diretamente da pré-visualização da página. Um ponto negativo desta imagem é a falta de explicação dos ícones utilizados, tornando o entendimento dos mesmos mais complicado.

Pontos que se destacaram com score abaixo da média foram a falta de opção para cancelar uma ação conforme a Figura \ref{fig:browsercms_erro_cancel} mostra. Outro problema apontado na Figura \ref{fig:browsercms_erro_msg_confirmacao} foi a mensagem de confirmação das ações executadas que aparecem muito rapidamente e estão  mal localizadas, dando a impressão que o usuário não completou determinada ação.

\begin{figure}[here]
\includegraphics[width=150mm]{images/browsercms_neutro_icones.jpg}
\caption{Edição de conteúdo em modo de pré visualização do BrowserCMS.}
\label{fig:browsercms_neutro_icones}
\end{figure}

\begin{figure}[here]
\includegraphics[width=150mm]{images/browsercms_erro_cancel.jpg}
\caption{Ausência de botão para cancelar ação no BrowserCMS}
\label{fig:browsercms_erro_cancel}
\end{figure}

\begin{figure}[here]
\includegraphics[width=150mm]{images/browsercms_erro_msg_confirmacao.jpg}
\caption{Mensagem de confirmação de ação aparece rapidamente e mal localizada no BrowserCMS}
\label{fig:browsercms_erro_msg_confirmacao}
\end{figure}



\subsection{Comentário da Inspeção por Checklist}

Pelos resultados gerados podemos observar que o Refinery se saiu melhor, mas ainda segundo a avaliação com um score baixo, indicando que há algumas áreas a se melhorar a usabilidade. A ferramenta Radiant falhou em quesitos como busca e o resultado refletiu esta falha do software trazendo o pior score entre os 3. O BrowserCMS teve um desempenho intermediário entre os avaliados, mas com um score baixo em algumas áreas indica as melhorias que poderiam ser tomadas para uma melhor usabilidade.   
