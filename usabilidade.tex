\chapter{Usabilidade}

Por muitos anos, as interfaces encontradas, eram de difícil manuseio e confusas o que repelia parte de seus usuários, muitos quando se deparavam com tais ferramentes não sentiam-se confortáveis e abandonavam.

Mas, com o alto crescimento de usuários das ferramentes web, viu-se a necessidade de criar CMS mais simples que facilitassem a criação e manutenção de sites, pois o usuário precisa conseguir utilizar e desejar usar novamente, e isso só acontecerá se ele encontrar o que procura com facilidade.

\begin{quote} 
Se a interface com o usuário for muito rígida, lenta, desagradável, pessoas se sentem frustradas, abandonam e esquecem o produto.
\cite{usability_evaluation_learning}\footnote{Tradução livre do autor}
\end{quote}

A norma da International Organization for Standardization 9241 define usabilidade como "É a medida pela qual um produto pode ser usado por usuários específicos para alcançar objetivos específicos com efetividade, eficiência e satisfação em um contexto de uso específico"

\begin {quote}
A usabilidade visa impactar positivamente sobre o retorno do investimento para a empresa. Ela será argumento de vendas, passará uma imagem de qualidade, evitará prejuízos para os clientes, ligados ao trabalho adicional e ao “retrabalho” de correções freqüentes, por exemplo. A empresa desenvolvedora economizará custos de manutenção e de revisões nos produtos, como mostra o texto sobre Engenharia de Usabilidade.
\cite{nielsen_usabilidade}
\end{quote}

Sistemas que visam uma boa usabilidade tem como dever fomentar a criação de interfaces simples, de modo a não dificultar os processos, ajudando o usuário a ter controle de todo o ambiente sem ser obstrusiva.

Projetar visando a usabilidade envolve estabelecer os requisitos de usuários para um novo sistema ou produto, desenvolver soluções de desing, protótipos do sistema e da interface com o usuário, e testar com os usuários significativos. No entanto, antes de qualquer atividade de projeto ou avaliação de usabilidade começar, é necessário compreender o Contexto de uso do produto, exemplificando, os objetivos da comunidade de usuários, o usuário principal, tarefas e as característias do ambiente em que o sistema irá operar.
\cite{maguire_context_of_use}\footnote{Tradução livre do autor}

Interfaces com baixa qualidade de uso trazem diversos problemas, dentre os quais:
confusão aos usuários, treinamento excessivo, desmotivação a exploração, indução ao
erro, entre outras. Estes problemas podem ser detectados por diversos métodos de
avaliação, realizados ao longo do processo de desenvolvimento. Os métodos de
avaliação mais utilizados se concentram em avaliar a usabilidade de um sistema.
\cite{maguire_context_of_use}\footnote{Tradução livre do autor}


A usabilidade pode ser dividida em cinco princípios básicos 

\begin{itemize}
  \item Intuitividade: O sistema deve apresentar facilidade de uso permitindo que, mesmo um usuário sem experiência, seja capaz de produzir algum trabalho satisfatoriamente.
  \item Eficiência: O sistema deve ser eficiente em seu desempenho apresentando um alto nível de produtividade.
  \item Memorização: Suas telas devem apresentar facilidade de memorização permitindo que usuários ocasionais consigam utilizá-lo mesmo depois de um longo intervalo de tempo.
  \item Erro: A quantidade de erros apresentados pelo sistema deve ser o mais reduzido possível, além disso, eles devem apresentar soluções simples e rápidas mesmo para usuários iniciantes. Erros graves ou sem solução não podem ocorrer.
  \item Satisfação: O sistema deve agradar ao usuário, sejam eles iniciantes ou avançados, permitindo uma interação agradável.
\end{itemize}
\cite{nielsen_usabilidade}


A partir destes princípios, é possivel aprofunda-los e especializar em alguns critérios que podem ser avaliados entre diferentes métodos de avaliação.

Existem algumas maneiras de se avaliar a usabilidade e ergonomia de um sistema, explorando os critérios a partir dos princípios acima descritos.

\section{Métodos de Avaliação}

Se a interface não tem uma boa qualidade, muitos problemas podem ser desencadeados. As mesmas podem ser confusas ao usuário, o que gerará muito treinamento, também a desmotivação em explorar as ferramentas o que induzirá ao erro, entre outros aspectos.
Esses problemas podem ser detectados por diversos métodos de avaliação, que são realizados ao longo do processo de desenvolimento, esses concentram-se em avaliar a usabilidade do sistema.
\cite{desigining_user_interface}


Avaliação de Usabilidade é um nome genérico para um conjunto de métodos que são baseados em avaliadores inspecionando a interface. Tipicamente, inspeção de usabilidade é direcionado a procurar problemas de usabilidade em uma interface. Vários métodos de inspeção focam nas especificações de interface com o usuário, que podem não estar necessariamente implementada, isso significa que a inspeção pode ser feita em estágios primários do ciclo de vida da engenharia de usabilidade. \cite{nielsen_95}

Há 3 principais técnicas de avaliação de ergonomia. São elas:

\begin{itemize}
  \item Técnicas Prospectivas
  \item Técnicas Preditivas ou Diagnósticas
  \item Técnicas Objetivas ou Empíricas
\end{itemize}

\begin{figure}[here]
\includegraphics[width=120mm]{tecnicas_usabilidade.jpg}
\caption{Diagrama das tecnicas de usabilidade}
\label{fig:tecnicas_usabilidade}
\end{figure}

\subsection{Técnicas Prospectivas}

Este tipo de técnica está baseada na aplicação de questionários/entrevistas com o usuário para avaliar sua satisfação ou insatisfação em relação ao sistema e sua operação. Ela mostra-se bastante pertinente na medida em que é o usuário a pessoa que melhor conhece o software, seus defeitos e qualidades em relação aos objetivos em suas tarefas. Nada mais natural em buscar suas opiniões para orientar revisões de projeto.

Muitas empresas de software elaboram e aplicam regularmente este tipo de questionário, como parte de sua estratégia de qualidade, porém constatou-se que os questionários de satisfação têm uma taxa de devolução reduzida (máximo 30\% retornam), o que indica a necessidade de elaboração de um questionário mais dinâmico, com questões mais sucintas e diretas e que tenham espaço para opiniões e sugestões, pois questionários longos, tornam-se cansativos.
\cite{cybil_apostila}

As técnicas prospectivas mais comuns são: questionários de opinão dos usuários, registros de uso do sistema e coleta de opiniões de especialistas. 

\subsection{Técnicas Preditivas ou Diagnósticas}

Técnicas preditivas são baseadas em avaliações de especialistas, e na competência destes avaliadores. 

As técnicas diagnósticas dispensam a participação direta de usuários nas avaliações, que se baseiam em verificações e inspeções de versões intermediárias ou acabadas de software interativo, feitas pelos projetistas ou por especialistas em usabilidade.
\cite{cybil_apostila}

As avaliações podem ser divididas em: 

\subsection{Avaliações Analíticas}

Essa técnica é empregada nas primeiras etapas da concepção de interfaces humano-computador, quando ela não passa de uma descrição da organização das tarefas interativas. Mesmo nesse nível, já é possível verificar questões como a consistência, a carga de trabalho e o controle do usuário sobre o diálogo proposto e a realizar. A especificação da futura tarefa interativa, pode ser realizada nos termos de um formalismo apropriado como MAD, GOMS (Goals, Operators, Methods and Selections rules) e CGL (Command Grammar Language).

Em particular, GOMS propõe uma tabela associando tempos médios de realização aos métodos primitivos, que correspondem as primitivas ações físicas ou cognitivas. Com base na descrição da tarefa realizada segundo o formalismo é possível calcular os tempos prováveis para a realização das tarefas previstas.
\cite{cybil_apostila}

As avaliações analíticas se dividem em métodos formais e aproximados.O método formal exige inspeção cuidadosa de seqüências de ação que um usuário realiza para concluir uma tarefa. Isto também é chamado "keystrokelevel analysis". Isto envolve dividir a tarefa em ações individuais, como movimentar o mouse para o menu ou o digitar no teclado e calcular o tempo que leva estas ações. 

O método aproximado é menos detalhado e provém resultados menos precisos, porém podem ser efetuados de maneira muito mais rápida. Envolve um processo similar de sequencia de ações que um usuário realiza nos quesitos físico,cognitivo e perceptivo. 

As vantagens de avaliações analíticas estão em uma predição precisa do quanto tempo leva a execução de uma tarefa e uma análise profunda do comportamento do usuário durante a mesma.

As desvantanges são o tempo e custo disto e ainda requerem uma avaliadores de alta capacidade.
\cite{andreas_holzinger}

\paragraph{Avaliações Heurísticas}

Uma avaliação heurística é um método de análise de interfaces e qualidades ergonômicas das interfaces humano-computador, com base no conjunto de critérios de usabilidade.  Os princípios são chamados de heurísticas pois são desenvolvidos a partir de uma série de experiências prévias, sintetizando pontos recorrentes. Essa avaliação é realizada por especialistas em ergonomia, baseados em sua experiência e competência no assunto. Eles examinam o sistema interativo e diagnosticam os problemas ou as barreiras que os usuários provavelmente encontrarão durante a interação.
\cite{cybil_apostila}

Avaliação heurística é o mais informal de métodos de inspecção. Ela é formada por um pequeno conjunto de especialistas, para analisar a aplicação através de uma lista de princípios de usabilidade reconhecidos - a heurística. Esta técnica é
parte dos métodos de usabilidade chamados de "discount". Ela é um método muito eficiente de engenharia de usabilidade,com uma relação custo-benefício alta.

Jakob Nielsen sugere um conjunto de 10 regras heurísticas para guiar uma avaliação, são elas:

\begin{itemize}
  \item Diálogos Simples e Naturais: O sistema deve apresentar facilidade de uso permitindo que, mesmo um usuário sem experiência, seja capaz de produzir algum trabalho satisfatoriamente.
  \item Eficiência: O sistema deve ser eficiente em seu desempenho apresentando um alto nível de produtividade.
  \item Memorização: Suas telas devem apresentar facilidade de memorização permitindo que usuários ocasionais consigam utilizá-lo mesmo depois de um longo intervalo de tempo.
  \item Erro: A quantidade de erros apresentados pelo sistema deve ser o mais reduzido possível, além disso, eles devem apresentar soluções simples e rápidas mesmo para usuários iniciantes. Erros graves ou sem solução não podem ocorrer.
  \item Satisfação: O sistema deve agradar ao usuário, sejam eles iniciantes ou avançados, permitindo uma interação agradável.
\end{itemize}



\paragraph{Inspeções por Checklist}

    




\subsection{Técnicas Objetivas ou Empíricas}












