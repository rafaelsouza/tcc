\chapter{cms}

CMS é a sigla para Content Management System, ou sistema gerenciador de conteúdo. É a idéia de um gerenciamento da informação de organizações que produzem muito conteúdo. São baseados na web, auxiliando em vários aspectos a publicação de conteúdo, desde sua forma de apresentação ao seu controle de versão. Estes CMS servem para a publicação de conteúdo,gerenciamento de transação de e-commerces, Wikis, gerenciamento de documentos, entre outras atividades. Sistema de Gerenciamento de Conteúdo podem de maneira simples e óbvia serem definidos por sua sigla, um sistema que gerencia conteúdos. Para uma melhor definição de CMS - Content Management Systems, o assunto será abordado a teoria de conteúdo e gestão dos mesmos.

\emph{Podemos dizer que um CMS é um framework, "um esqueleto" de website pré-programado, com recursos básicos e de manutenção e administração já prontamente disponíveis. É um sistema que permite a criação, armazenamento e administração de conteúdo de forma dinâmica, através de uma interface de usuário via Internet. Um CMS permite que a empresa tenha total autonomia sobre o conteúdo e evolução da sua presença na internet e dispense a assistência de terceiros ou empresas especializadas para manutenções de rotina. Nem mesmo é preciso um funcionário dedicado (webmaster), pois cada membro da equipe poderá gerenciar o seu próprio conteúdo, diminuindo os custos com recursos humanos. A habilidade necessária para trabalhar com um sistema de gerenciamento de conteúdo não vai muito além dos conhecimentos necessários para um editor de texto.} 
\cite{navita}

Linguisticamente, (CMS) significa qualquer sistema que auxilia no gerenciamento de conteudo - criação, armazenamento, indexação, arquivamento, publicação e distribuição do conteúdo.
\cite{what_is_cms}\footnote{Tradução livre do autor}

\subsection{O que é Conteúdo}

Há várias definições para Conteúdo, no contexto de CMS, uma visão mais simples de conteúdo vem de \cite{ecm_paper} "O termo conteúdo representa qualquer conteúdo eletrônico, incluindo registros,dados e metadados, como também documentos e websites". 
A definição de \cite{cms_bible} "Conteúdo,protanto, é a informação que você rotula com dados para entâo um computador porder organizar e sistematizar a coleção,gerenciamento e publicação". Estas definições apresentam algumas diferenças, mas nota-se que sem o auxílio de uma ferramenta específica para gerenciar conteúdos, seria uma tarefa deveras árdua.

\subsection{Tipos de CMS}

Como CMS é um conceito amplo, existem várias classificações entre os Gerenciadores de Conteúdo. Alguns podem ser mais específicos como para o funcionamento de blogs ou wikis, outros mais generalistas como os Web Content Management Systems (WCMS).

Segundo \cite{choosing_open_source_cms} eis as seguintes classificações

\subsubsection{Portais ou CMS genéricos ou WCMS} 

Estes CMS são muito populares, geralmente encontrados na confecção de sites corporativos, de pequeno até grande porte no acso de Portais. Eliminam em grande parte a complexidade de o site ser administrado por uam pessoa técnica, conhecida como webmaster. Com este tipo de CMS pessoas com pouco conhecimento técnico podem publicar conteúdo.

\paragraph{Principais Características}

\begin{itemize}
  \item Criar a gerenciar seções de conteúdos.
  \item Criar páginas e adicionar conteúdos de textos ou imagens.
  \item Editar conteúdo publicado.
  \item Administração por múltiplos usuários.
  \item Versionamento de conteúdo.
  \item Gerenciamento de Workflow.
\end{itemize}

\paragraph{Exemplos de WCMS} 

\begin{itemize}
  \item BrowserCMS
  \item RadiantCMS
  \item RefineryCMS
  \item ZenaCMS
  \item Drupal
  \item Joomla
  \item Liferay
\end{itemize}  


\subsubsection{Blog CMS} 

Blogs também são considerados CMS pois são autorais, publicam conteúdo de texto,imagem,vídeos.

\paragraph{Principais Características}

\begin{itemize}
  \item Criar posts.
  \item Categorizar Posts.
  \item Gerenciar comentários.
  \item Adicionar imagens,videos, textos.
\end{itemize}

\paragraph{Exemplos de Blog CMS} 

\begin{itemize}
  \item WordPress
  \item Mephisto
  \item Typo
  \item Blogger
\end{itemize}

\subsubsection{Wiki CMS} 

\emph{Wiki é uma página ou coleção de páginas web projetadas para permitir o acesso, a qualquer usuario, para contribuir ou modificar o conteúdo (excluíndo os usuários bloqueados), utilizando uma linguagem de marcação simplificada. Wikis são geralmente usados para criarem sites colaborativos e amplificar sitems de comunidades.}
\cite{choosing_open_source_cms}{página 22}\footnote{Tradução livre do autor} 


\paragraph{Principais Características}

\begin{itemize}
  \item Facilidade de criar páginas, chamdas de wikiweb.
  \item Linguagem de marcação simples.
  \item Criação de links automatizadas, ainda que o link ainda não exista.
  \item Sistema completo de versionamento.
  \item Pode ter acesso restrito à usuarios ou grupos.
\end{itemize}

\paragraph{Exemplos de Wiki CMS} 

\begin{itemize}
  \item MediaWiki
  \item TWiki
  \item Typo
  \item Blogger
\end{itemize}


\subsubsection{eLearning CMSs} 

eLearning CMS são gerenciadores de conteúdos especializados em ensino a distância. Também conhecidos como LMS Learning Management Systems, eles tem responsabilidade pela administração,documentação,registro e relatório de cursos.

\paragraph{Principais Características}

\begin{itemize}
  \item Gerenciar cursos, estudantes, professores.
  \item Criar um curso e um programa de aprendizado.
  \item Criar documentos, testes, discussões e anúncios.
  \item Sistema de chat, forum, blogs, etc.
\end{itemize}

\paragraph{Exemplos de eLearning CMS} 

\begin{itemize}
  \item Dokeos
  \item Moodle
  \item LAMS
\end{itemize}


\cite{choosing_open_source_cms}\footnote{Tradução livre do autor}

\subsection{Ferramentas Avaliadas}

As ferramentas avaliadas pertencem ao grupo dos Web Content Management Systems, e dentro deste contexto elas serão avaliadas.

\subsubsection{RadiantCMS}

\emph{Radiant é um sistema de gerenciamento de conteúdo aberto, simples e projetado para pequenas equipes}
\cite{radiant_website}\footnote{Tradução livre do autor}

Radiant é um dos CMS mais antigos disponíveis para Ruby, lançado em 2006. De instalação e uso simples tem uam grande quantidade de plugins disponíveis e uma grande comunidade mantenedora da aplicação.

\paragraph{Características}

\begin{itemize}
  \item Interface elegante intuitiva
  \item Bom sistema de extensões e plugins.
  \item Sistema de usuários e permissões simples.
\end{itemize}

Radiant 

















