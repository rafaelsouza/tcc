\chapter{CMS}

CMS é a sigla para Content Management System, ou sistema gerenciador de conteúdo. É a idéia de um gerenciamento da informação de organizações que produzem muito conteúdo. São baseados na web, auxiliando em vários aspectos a publicação de conteúdo, desde sua forma de apresentação ao seu controle de versão. Estes CMS servem para a publicação de conteúdo, gerenciamento de transação de e-commerces, Wikis, gerenciamento de documentos, entre outras atividades. Sistemas de Gerenciamento de Conteúdo podem de maneira simples e óbvia serem definidos por sua sigla, um sistema que gerencia conteúdos. Para uma melhor definição de CMS - Content Management Systems, o assunto será abordado na teoria de conteúdo e gestão dos mesmos.

\emph{Podemos dizer que um CMS é um framework, 'um esqueleto' de website pré-programado, com recursos básicos e de manutenção e administração já prontamente disponíveis. É um sistema que permite a criação, armazenamento e administração de conteúdo de forma dinâmica, através de uma interface de usuário via Internet. Um CMS permite que a empresa tenha total autonomia sobre o conteúdo e evolução da sua presença na internet e dispense a assistência de terceiros ou empresas especializadas para manutenções de rotina. Nem mesmo é preciso um funcionário dedicado (webmaster), pois cada membro da equipe poderá gerenciar o seu próprio conteúdo, diminuindo os custos com recursos humanos. A habilidade necessária para trabalhar com um sistema de gerenciamento de conteúdo não vai muito além dos conhecimentos necessários para um editor de texto.} \cite{navita}

Linguisticamente, (CMS) significa qualquer sistema que auxilia no gerenciamento de conteúdo - criação, armazenamento, indexação, arquivamento, publicação e distribuição do conteúdo.
\cite{what_is_cms}\footnote{Tradução livre do autor}

Em suma, o grande diferencial de um CMS é permitir que o conteúdo de um website possa ser modificado de forma rápida e segura de qualquer computador conectado à Internet. Um sistema de gerenciamento de conteúdo reduz custos e ajuda a suplantar barreiras potenciais à comunicação web reduzindo o custo da criação, contribuição e manutenção de conteúdo.
\cite{navita}

\section{O que é Conteúdo}

Há várias definições para Conteúdo, no contexto de CMS, uma visão mais simples de conteúdo vem de \cite{ecm_paper} \emph{O termo conteúdo representa qualquer conteúdo eletrônico, incluindo registros, dados e metadados, como também documentos e websites}. 
A definição de \cite{cms_bible} \emph{Conteúdo, portanto, é a informação que você rotula com dados para então um computador poder organizar e sistematizar a coleção, gerenciamento e publicação}. Estas definições apresentam algumas diferenças, mas ajudam a modelar o conceito de conteúdo dentro do contexto do trabalho.

\section{Tipos de CMS}

Como CMS é um conceito amplo, existem várias classificações entre os Gerenciadores de Conteúdo. Alguns, podem ser mais específicos como o funcionamento de blogs ou wikis, outros mais generalistas como os Web Content Management Systems (WCMS).

Segundo \cite{choosing_open_source_cms} eis as seguintes classificações

\subsection{Portais ou CMS genéricos ou Web Content Management Systems} 

Estes CMS são muito populares, geralmente encontrados na confecção de sites corporativos, de pequeno até grande porte no caso de Portais. Eliminam em grande parte a complexidade do site ser administrado por uma pessoa técnica, conhecida como webmaster. Com este tipo de CMS pessoas com pouco conhecimento técnico podem publicar conteúdos.

\paragraph{Principais Características}

\begin{itemize}
  \item Criar a gerenciar seções de conteúdos.
  \item Criar páginas e adicionar conteúdos de textos ou imagens.
  \item Editar conteúdo publicado.
  \item Administração por múltiplos usuários.
  \item Versionamento de conteúdo.
  \item Gerenciamento de Workflow.
\end{itemize}

\paragraph{Exemplos de WCMS} 

\begin{itemize}
  \item BrowserCMS\footnote{http://www.browsercms.org}
  \item RadiantCMS\footnote{http://www.radiantcms.org}
  \item RefineryCMS\footnote{http://www.refinerycms.com}
  \item ZenaCMS\footnote{http://www.zenadmin.org}
  \item Drupal\footnote{http://www.drupal.org}
  \item Joomla\footnote{http://www.joomla.org}
  \item Liferay\footnote{http://www.liferay.com}
\end{itemize}  


\subsection{Blog CMS} 

Blogs também são considerados CMS pois são autorais, publicam conteúdo de texto, imagem e vídeos. Ele é formado por um conjunto de entradas (posts) do(s) autor(es), que ainda podem receber comentários de seus leitores.

\paragraph{Principais Características}

\begin{itemize}
  \item Criar posts.
  \item Categorizar Posts.
  \item Gerenciar comentários.
  \item Adicionar imagens, vídeos e textos.
\end{itemize}

\paragraph{Exemplos de Blog CMS} 

\begin{itemize}
  \item WordPress\footnote{http://www.wordpress.com}
  \item Mephisto\footnote{http://www.mephistoblog.com/}
  \item Typo\footnote{http://typosphere.org/}
  \item Blogger\footnote{http://www.blogger.com/}
\end{itemize}

\subsection{Wiki CMS} 

\emph{Wiki é uma página ou coleção de páginas web projetadas para permitir o acesso, a qualquer usuário, para contribuir ou modificar o conteúdo (excluindo os usuários bloqueados), utilizando uma linguagem de marcação simplificada. Wikis são geralmente usados para criarem sites colaborativos e ampliar sitemas de comunidades.}
\cite{choosing_open_source_cms}\footnote{Tradução livre do autor} 


\paragraph{Principais Características}

\begin{itemize}
  \item Facilidade de criar páginas, chamadas de wikiweb.
  \item Linguagem de marcação simples.
  \item Criação de links automatizados, mesmo que o link ainda não exista.
  \item Sistema completo de versionamento.
  \item Pode ter acesso restrito à usuários ou grupos.
\end{itemize}

\paragraph{Exemplos de Wiki CMS} 

\begin{itemize}
  \item MediaWiki\footnote{http://www.mediawiki.org/}
  \item TWiki\footnote{http://www.twiki.org/}
  \item Confluente\footnote{http://www.atlassian.com/software/confluence/}
  \item DokuWiki\footnote{http://www.dokuwiki.org/}
\end{itemize}


\subsection{eLearning CMSs} 

eLearning CMS são gerenciadores de conteúdos especializados em ensino à distância. Também conhecidos como LMS Learning Management Systems, eles têm responsabilidade pela administração, documentação, registro e relatório de cursos.
 
\paragraph{Principais Características}

\begin{itemize}
  \item Gerenciar cursos, estudantes, professores.
  \item Criar um curso e um programa de aprendizado.
  \item Criar documentos, testes, discussões e anúncios.
  \item Sistema de chat, forum, blogs, etc.
\end{itemize}

\paragraph{Exemplos de eLearning CMS} 

\begin{itemize}
  \item Dokeos\footnote{http://www.dokeos.com/}
  \item Moodle\footnote{http://www.moodle.org/}
  \item LAMS\footnote{http://www.lamsinternational.com/}
\end{itemize}


