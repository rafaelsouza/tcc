\chapter{Trabalhos relacionados}

Na pesquisa bibliográfica, foram encontrados alguns trabalhos relacionados à avaliações de ferramentas de gerenciamento de conteúdo. 

Yan \cite{search_content_management}, realizou em seu trabalho uma avaliação detalhada entre três ferramentas, nos aspectos de preservação de dados, metadados, acesso ao conteúdo, e funcionalidades requeridas pela biblioteca da Universidade do Arizona.

A Athos Origin \cite{qsos_site} publicou algumas avaliações de cms feitas sobre a metodologia que eles propõe para avaliar software de código aberto. Estas avaliações foram feitas sob os aspectos da maturidade da ferramenta e dos desenvolvedores dela, distribuição, funcionalidades, arquitetura, compatibilidade entre browsers, entre outros quesitos. 

Michelinakis \cite{michelinakis_cms_evaluation}, fez uma avaliação de sete ferramentas CMS open source em seu trabalho, focando requisitos e funcionalidades no contexto comercial.

Há ainda outros sites que comparam ferramentas de CMS, como o CMS Matrix\footnote{http://www.cmsmatrix.org}. Este site compara os CMS de acordo um uma série de critérios como requisitos do sistema, funcionalidade, distribuição, plataforma suportada, repositório de dados suportados, entre outros. A falta de padronização dos dados é um fator que dificulta a comparação, pois algumas ferramentas estão com informações incompletas. As informações sobre ferramentas no site é fomentada com dados oriundos dos próprios usuários cadastrados.

Não foram encontrados trabalhos relacionados focados em avaliação de cms com aspecto da usabilidade. As palavras chave utilizadas foram "CMS evaluation usability", "CMS assesment usability", "Content Management Systems usability assesment", "Content Management Systems usability evaluation". As bases de pesquisa utilizadas foram, ACM Digital Library, Wiley Online Library, Cambridge Journals Online, SpringerLink (MetaPress), Oxford Journals, IEEE Xplore, Web of Science.


   

 
