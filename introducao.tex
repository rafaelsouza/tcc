\chapter{Introdução}

A internet vem se popularizando cada vez mais, com isso a quantidade de conteúdo e informações digitais cresce de forma significativa. Segundo o ISC o crescimento de domínios registrados é muito grande, chegando próximo aos 800 mil domínios. Muita gente com pouco conhecimento esta publicando suas informações e gerando novos conteúdos.

\begin{figure}[here]
\includegraphics[width=150mm]{images/isc_hosts.png}
\caption{Número de domínios registrados dos últimos anos}
\label{fig:isc_hosts.png}
\end{figure}

Diante desses fatos,  é intrínseco a necessidade de gerenciar e publicar esses conteúdos de forma rápida, tanto por indivíduos como organizações. A partir desta necessidade, surgem os CMS (Content Management Systems) que proporcionam versatilidade, quando o objetivo é produzir de forma rápida e organizada, soluções integradas para a exibição de conteúdo na internet. 

Na escolha de um CMS para desenvolvimento deve-se pesar também não só a ferramentas mas também a plataforma a qual é desenvolvida. Neste trabalho o foco será nas ferramentas da plataforma Ruby on Rails, que está se tornando muito popular nos últimos anos.

Para efetuar uma escolha de uma ferramenta, a usabilidade tem recebido maior atenção, pois foi reconhecida como uma propriedade fundamental no sucesso das aplicações web.

%Definir os métodos para a usabilidade é, portanto, uma das metas atuais de pesquisas. Além disso, as empresas tem reconhecido a importância dos métodos de usabilidade durante o processo de desenvolvimento, para verificar a usabilidade das aplicações, antes da sua implantação.

Outro aspecto relevante na escolha de um CMS, é caso ela for uma ferramenta de código aberto, é a sua comunidade de desenvolvimento. Avaliar a comunidade em torno da ferramenta é um dos parâmetros deste trabalho.

%Alguns estudos têm demonstrado, de fato, como o uso de tais métodos permite reduzir custos, em uma relação custo-benefício, pois reduzem a necessidade de mudanças após a entrega dos aplicativos.

%A satisfação do usuário final, ao atingir seus objetivos e resultados com eficiência, está relacionado diretamente a usabilidade e qualidade dos sistemas interativos. Principalmente se estes forem fáceis de serem utilizados e permitam de forma rápida criar e atualizar, também econômicos, pois o acesso a esses CMS são gratuitos.

%Fazendo uma analogia, percebe-se  que gerenciamento de conteúdos e usabilidade estão diretamente relacionados, pois a usabilidade é um fator fundamental e intrínseco para o sucesso e qualidade de um projeto e satisfação do usuário, que trará benefícios significativos e tangíveis para o sistema de conteúdo.


%A usabilidade foi escolhida devido a grande importância que o assunto tem entre as ferramentas CMS, tendo em vista que o intuito das ferramentas é sua utilização por usuários com pouca experiência em web, e com grande volume de dados para gerenciar. 

%O aspecto da comunidade open source veio para tentar mostrar qual a relevância que o projeto tem, ultimamente dentro da comunidade open source, pois ao efetuar a escolha de uma ferramenta deve  levar em consideração também qual a estabilidade da ferramenta, se os bugs são encontrados e  corrigidos frequentemente e se seus mantenedores são receptivos a conversas sobre o projeto para tirar dúvidas. 

Baseado nisto, este trabalho vai avaliar um subconjunto de ferramentas CMS, Radiant, Refinery, BrowserCMS, Zena, escritas em Ruby on Rails de acordo com os parâmetros escolhidos.
