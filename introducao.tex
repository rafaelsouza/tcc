\chapter{Introdução}

A quantidade de conteúdo e informações digitais vem crescendo de forma significativa. Segundo o ISC o crescimento de dominios registrados é muito grande, chegando próximo ao bilhão de domínios. 

\begin{figure}[here]
\includegraphics[width=150mm]{images/isc_hosts.png}
\caption{Número de domínios registrados dos últimos anos}
\label{fig:isc_hosts.png}
\end{figure}

Diante desses fatos, é intrínseco a necessidade de gerenciar e publicar esses conteúdos de forma rápida, tanto por indivíduos como organizações. A partir desta necessidade, surgem os CMS (Content Management Systems) que proporcionam versatilidade, quando o objetivo é produzir de forma rápida e organizada, soluções integradas para a exibição de conteúdo na internet. 

Na escolha de um CMS para desenvolvimento deve-se pesar também não só a ferramentas mas também a plataforma a qual é desenvolvida, neste trabalho o foco será nas ferramentas da plataforma Ruby on Rails, que está se tornando muito popular nos ultimos anos.

A usabilidade, por sua vez com o passar dos anos, tem recebido maior atenção, pois foi reconhecida como uma propriedade fundamental no sucesso das aplicações web.

Definir os métodos para a usabilidade é, portanto, uma das metas atuais de pesquisas. Além disso, as empresas tem reconhecido a importância dos métodos de usabilidade durante o processo de desenvolvimento, para verificar a usabilidade das aplicações, antes da sua implantação.

Alguns estudos têm demonstrado, de fato, como o uso de tais métodos permite reduzir custos, em uma relação custo-benefício, pois reduzem a necessidade de mudanças após a entrega dos aplicativos.

A satisfação do usuário final, ao atingir seus objetivos e resultados com eficiência, está relacionado diretamente a usabilidade e qualidade dos sistemas interativos. Principalmente se estes forem fáceis de serem utilizados e permitam de forma rápida criar e atualizar, também econômico, pois o acesso a esses CMS são gratuitos.

Fazendo uma analogia, percebe-se  que gerenciamente de conteúdos e usabilidade estão diretamente relacionados, pois a usabilidade é um fator fundamental e intrínseco para o sucesso e qualidade de um projeto e satisfação do usuário, que trará benefícios significativos e tangíveis para o sistema de conteúdo.


