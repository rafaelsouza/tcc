\subsection{BrowserCMS}

\begin{figure}[here]
\includegraphics[width=130mm]{images/browsercms_result_checklist_table.jpg}
\caption{Tabela com o resultado da inspeção por checklist no BrowserCMS}
\label{fig:resultado_checklist_browsercms_tabela}
\end{figure}

Esta Figura \ref{fig:resultado_checklist_browsercms_tabela} contendo a tabela gerada a partir da checklist, destaca-se o alto índice no critérios de Navigation & IA. 

Os destaques negativos ficaram nos critérios da Search,Forms \& Data Entry,Help Feedback \& Error Tolerance. 

A Figura \ref{fig:resultado_checklist_browsercms_grafico} demonstra o resultado em forma de gráfico para uma melhor visualização.

\begin{figure}[here]
\includegraphics[scale=0.5]{images/browsercms_result_checklist_graph.jpg}
\caption{Gráfico com o resultado da inspeção por checklist no BrowserCMS}
\label{fig:resultado_checklist_browsercms_grafico}
\end{figure}

O BrowserCMS traz uma proposta de interface diferente de Radiant e Refinery, onde o usuário edita o conteúdo fazendo diretamente uma pré visualização do mesmo. Há uma curva de aprendizado para a ferramenta, mas após acostumado ela traz alguns benefícios como a possibilidade de reutilizar conteúdos de outras páginas, facilitando a manutenção do conteúdo.

Um ponto a se destacar, está na Figura \ref{fig:browsercms_neutro_icones} onde ilustra a possibilidade de editar o conteúdo diretamente da pré-visualização da página. Um ponto negativo desta imagem é a falta de explicação dos ícones utilizados, tornando o entendimento dos mesmos mais complicado.

Pontos que se destacaram com score abaixo da média foram a falta de opção para cancelar uma ação conforme a Figura \ref{fig:browsercms_erro_cancel} mostra. Outro problema apontado na Figura \ref{fig:browsercms_erro_msg_confirmacao} foi a mensagem de confirmação das ações executadas que aparecem muito rapidamente e estão  mal localizadas, dando a impressão que o usuário não completou determinada ação.

\begin{figure}[here]
\includegraphics[width=150mm]{images/browsercms_neutro_icones.jpg}
\caption{Edição de conteúdo em modo de pré visualização do BrowserCMS.}
\label{fig:browsercms_neutro_icones}
\end{figure}

\begin{figure}[here]
\includegraphics[width=150mm]{images/browsercms_erro_cancel.jpg}
\caption{Ausência de botão para cancelar ação no BrowserCMS}
\label{fig:browsercms_erro_cancel}
\end{figure}

\begin{figure}[here]
\includegraphics[width=150mm]{images/browsercms_erro_msg_confirmacao.jpg}
\caption{Mensagem de confirmação de ação aparece rapidamente e mal localizada no BrowserCMS}
\label{fig:browsercms_erro_msg_confirmacao}
\end{figure}

