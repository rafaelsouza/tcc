\section{Justificativa}
 
O crescimento e a expansão da internet, fomentaram a criação de ferramentas CMS. Muitas delas são desenvolvidas e distribuídas sob licensas livres em comunidades open source na internet. Diante deste cenário, um profissional de Sistemas de Informação deve avaliar entre as opções disponíveis e dependendo do contexto definir a utilização de uma ou outra ferramenta.

Este trabalho faz uma avaliação de três ferramentas CMS de código aberto escritas em Ruby on Rails,  abordando a usabilidade e a comunidade ao redor das ferramentas CMS, como parâmetros para a avaliação. Assim demonstrando uma pssibilidade de avaliação de ferramentas.





    
