\section{Justificativa}
 
O crescimento e a expansão da internet, fomentaram a criação de ferramentas CMS. Muitas delas são desenvolvidas e distribuídas sob licensas livres em comunidades open source na internet. Diante deste cenário, um profissional de Sistemas de Informação deve avaliar entre as opções disponíveis e, dependendo do contexto, definir a utilização de uma ou outra ferramenta. Assim o presente trabalho visa avaliar três ferramentas CMS de código aberto escritas em Ruby on Rails,  de acordo com a usabilidade para o usuario especializado.

% e a comunidade ao redor das ferramentas CMS, como parâmetros para a avaliação. Assim demonstrando uma possibilidade de avaliação de ferramentas.





    
