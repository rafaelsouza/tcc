\chapter{Metodologia}

Este capítulo se dedica a informar quais as serão os métodos de avaliação entre as ferramentas CMS - BrowserCMS, Radiant, Refinery.  

\section{QSOS - Qualification and Selection of Open Source Software}

\emph{Qualification and Selection of Open Source Sofware} é uma metodologia de avaliação e seleção de software livre. O método encontra-se na versão 1.6, consiste em uma série de 4 passos independentes a seguir \cite{qsos_site} 

\begin{description} 
    \item[Definição] Definição dos projetos a serem avaliados.
    \item[Avaliação] Avaliação dos mesmos segundo os critérios providos pelo QSOS, como funções do sistema,riscos ao usuário e riscos ao provedor do serviço.
    \item[Qualificação] Determinação dos pesos nos critérios de avaliação, modelando o contexto.
    \item[Seleção] Seleção das melhores no passo 3 com os dados  oriundos dos passos 1 e 2.
\end{description}

\begin{figure}[here]
\includegraphics[width=150mm]{images/4_steps.png}
\caption{Quatro etapas do processo QSOS}
\label{fig:qsos_steps.png}
\end{figure}

Os critérios de avaliação da metodologia QSOS são subdivididos em 2 grandes áreas, uma mais genérica e outra mais específica. Os critérios genéricos lidam com assuntos comuns a todos os sofwares enquanto os específicos lidam com peculiaridades que cada categoria de software tem \cite{openbrr_qsos}. A avaliação é dada em scores de 0 1 e 2, conforme o exemplo da tabela \ref{table:qsos_avaliacao}  

\begin{table}[ht]
\caption{Tabela de exemplo dos critérios da QSOS } % title of Table
\centering % used for centering table
\begin{tabular}{c | c | p{4.2cm} | p{4.2cm}} % centered columns (5 columns)
\hline\hline %inserts double horizontal lines
Critério & Score 0 & Score 1 & Score 2  \\ [0.5ex] % inserts table
% heading
\hline % inserts single horizontal line
Age        & less than 3 months  & if between 3 months and 3 years   & after 3 years    \\ \hline
Popularity & Very few users identified  & Detectable use on Internet & Numerous users, numerous references    \\ \hline
References & None   & Few refences, non critical usages  & Often implemented for critical applications \\ [1ex] % [1ex] adds vertical space
\hline % inserts single line
\end{tabular}
\label{table:qsos_avaliacao} % is used to refer this table in the text
\end{table}

Há uma versão de uma avaliação QSOS para ferramentas CMS, disponível em: \url{http://www.qsos.org/templates/cms.html}. Conforme abordado nas limitações, este trabalho fará os critérios genéricos e os critérios de gerenciamento de conteúdo web. Os critérios de avaliação utlizados por esta neste trabalho se encontram em anexo. As avaliações serão feitas pelo autor.

\section{Inspeção por Checklist}

Este método foi selecionado, pois pode ser aplicado por uma pessoa sem muita experiência com usabilidade, como o próprio autor deste trabalho, nas ferramentas CMS selecionadas. Os critérios da checklist são baseados em heurísticas

A inspeção adotada é uma checklist mais voltada a websites, adaptadas as necessidades das ferramentas avaliadas. A inspeção foi retirada do site \url{http://www.userfocus.co.uk/resources/navchecklist.html}. Os critérios avaliados, se encontram em anexo. O autor é quem fará as inspeções nas ferramentas.


\section{Avaliação da comunidade}

Devido as soluções serem open source, uma análise sobre a comunidade em torno da ferramenta é um aspecto importante na avaliação das ferramentas. Como todas as ferramentas possuem seu código disponibilizado dentro do github.com, o próprio site fornece números interessantes sobre os projetos, adicionados a estes números serão avaliados também números da lista de discussão de email e os canais de comunicação do projeto.

Os critérios para avaliar serão:

\begin{description}
    \item[Watchers] - Número de pessoas que acompanham o projeto (novos commits), chamados \emph{watchers}
    \item[Forks] - Número de versões de terceiros em paralelo do projeto (fork).
    \item[Commits] - Número de commits no ano de 2010, até 12 de novembro.
    \item[Contribuidores] - Número de contribuidores do projeto.
    \item[Pageviews] - Número de pageviews na página do projeto entre agosto - novembro 2010.
    \item[Lista de email] - Possui lista de discussão por email?
    \item[Número pessoas lista] - Número de pessoas cadastradas na lista de discussão por email. 
    \item[Número mensagens lista] - Número de mensagens postadas no ano de 2010 na lista de email.
    \item[Twitter] - Possui Twitter?
    \item[Seguidores twitter] - Quantos seguidores possui no Twitter?
    \item[IRC] - Possui canal no IRC (Internet Relay Chat)?
\end{description}



