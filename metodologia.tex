\chapter{Metodologia}

Este capítulo se dedica a informar quais as serão os métodos de avaliação entre as ferramentas CMS.  

\section{Inspeção por Checklist}

Este método foi selecionado dentre os apresentados anteriormente, pois pode ser aplicado por um não expert, como o próprio autor nas ferramentas selecionadas.

A inspeção adotada é uma checklist mais voltada a websites, adaptadas as necessidades das ferramentas avaliadas. A inspeção foi retirada do site \emph{http://www.userfocus.co.uk/resources/navchecklist.html} e se encontra em anexo.

\section{Avaliação da comunidade}

Devido as soluções serem open source, uma análise sobre a comunidade em torno da ferramenta é um aspecto importante na avaliação das ferramentas. Como todas as ferramentas possuem seu código disponibilizado dentro do github.com, o próprio site fornece números interessantes sobre os projetos, adicionados a estes números serão avaliados também numeros da lista de discussão de email e os canais de comunicação do projeto.

Os critérios para avaliar serão:

\begin{description}
    \item[Watchers] - Número de pessoas que acompanham o projeto (novos commits), chamados \emph{watchers}
    \item[Forks] - Número de versões de terceiros em paralelo do projeto (fork).
    \item[Commits] - Número de commits no ano de 2010, até 12 de novembro.
    \item[Contribuidores] - Número de contribuidores do projeto.
    \item[Pageviews] - Número de pageviews na página do projeto entre agosto - novembro 2010.
    \item[Lista de email] - Possui lista de discussão por email ?
    \item[Número pessoas email] - Número de pessoas cadastradas na lista de discussão por email. 
    \item[Número mensagens lista] - Número de mensagens postadas no ano de 2010 na lista de email.
    \item[Twitter] - Possui Twitter ?
    \item[Seguidores twitter] - Quantos seguidores possui no Twitter ?
    \item[IRC] - Possui canal no IRC (Internet Relay Chat) ?
\end{description}



