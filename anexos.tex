\chapter{Anexos}

\section{Critérios QSOS}

Critérios utilizados para a avaliação de Qualification and Selection of Open Source software.

\begin{itemize}
    \item Web content management
    \begin{enumerate}
        \item Blog
            \begin{enumerate}
                \item The feature doesn't exist
                \item Work fine for at last odt files
                \item Work fine for the majority of the formats
            \end{enumerate}
        \item Recent changes advertised on the web interface
            \begin{enumerate}
                \item No such features
                \item Recent changes advertised according to the today's date
                \item Recent changes advertised only changes that happened since user's last visit
            \end{enumerate} 
        \item Support for other notifications media
            \begin{enumerate}
                \item This feature doesn't exist
                \item The feature exists for one communication media
                \item This feature exists for several media: e.mail, and IM for instance 
            \end{enumerate}               
    \end{enumerate}
    
\end{itemize}

\section{Checklist}

Checklist utilizada para efetuar a avaliação de usabilidade. Os critérios não foram traduzidos pois poderiam perder um pouco do significado.

\begin{itemize}
  \item home page usability guidelines
      \begin{enumerate}
        \item The items on the home page are clearly focused on users' key tasks ("featuritis" has been avoided).
        \item If the site is large, the home page contains a search input box.
        \item Useful content is presented on the home page or within one click of the home page.
        \item Links on the home page begin with the most important keyword (e.g. "Sun holidays" not "Holidays in the sun").
        \item There is a short list of items recently featured on the homepage, supplemented with a link to archival content.
        \item Navigation areas on the home page are not over-formatted and users will not mistake them for adverts.
        \item The value proposition is clearly stated on the home page (e.g. with a tagline or welcome blurb).
        \item Navigation choices are ordered in the most logical or task-oriented manner (with the less important corporate information at the bottom).
        \item The title of the home page will provide good visibility in search engines like Google.
        \item All corporate information is grouped in one distinct area (e.g. "About Us").
        \item Users will understand the value proposition.
        \item By just looking at the home page, the first time user will understand where to start.
        \item The home page shows all the major options.
        \item The home page of the site has a memorable URL.
        \item The home page is professionally designed and will create a positive first impression.
        \item The design of the home page will encourage people to explore the site.
        \item The home page looks like a home page; pages lower in the site will not be confused with it.
      \end{enumerate}
  \item Task Orientation
      \begin{enumerate}
        \item The site is free from irrelevant, unnecessary and distracting information.
        \item Excessive use of scripts, applets, movies, audio files, graphics and images has been avoided.
        \item The site avoids unnecessary registration.
        \item The critical path (e.g. purchase, subscription) is clear, with no distractions on route.
        \item Information is presented in a simple, natural and logical order.
        \item The number of screens required per task has been minimised.
        \item The site requires minimal scrolling and clicking.
        \item The site correctly anticipates and prompts for the user’s probable next activity.
        \item Users can complete common tasks quickly.
        \item The site makes the user’s work easier and quicker than without the system.
        \item The most important and frequently used topics, features and functions are close to the centre of the page, not in the far left or right margins.
        \item The user does not need to enter the same information more than once.
        \item Important, frequently needed topics and tasks are close to the 'surface' of the web site.
        \item Typing (e.g. during purchase) is kept to an absolute minimum, with accelerators ("one-click") for return users.
        \item The path for any given task is a reasonable length (2-5 clicks).
        \item When there are multiple steps in a task, the site displays all the steps that need to be completed and provides feedback on the user’s current position in the workflow.
        \item The use of metaphors is easily understandable by the typical user.
        \item Data formats follow appropriate cultural conventions (e.g. miles for UK).
        \item Details of the software's internal workings are not exposed to the user.
        \item The site caters for users with little prior experience of the web.        
        \item A typical first-time visitor can do the most common tasks without assistance.
        \item When they return to the site, users will remember how to carry out the key tasks.
        \item Action buttons (such as "Submit") are always invoked by the user, not automatically invoked by the system when the last field is completed.
        \item Command and action items are presented as buttons (not, for example, as hypertext links).
        \item If the user is half-way through a transaction and quits, the user can later return to the site and continue from where he left off.
        \item When a page presents a lot of information, the user can sort and filter the information.
        \item If there is an image on a button or icon, it is relevant to the task.
        \item The site prompts the user before automatically logging off the user, and the time out is appropriate.
        \item The site is robust and all the key features work (i.e. there are no javascript exceptions, CGI errors or broken links).
        \item The site supports novice and expert users by providing different levels of explanation (e.g. in help and error messages).
        \item The site allows the user to customise operational time parameters (e.g. time until automatic logout).
      \end{enumerate}
  \item Navigation and IA
      \begin{enumerate}
        \item There is a convenient and obvious way to move between related pages and sections and it is easy to return to the home page.
        \item The information that users are most likely to need is easy to navigate to from most pages.
        \item Navigation choices are ordered in the most logical or task-oriented manner.
        \item The navigation system is broad and shallow (many items on a menu) rather than deep (many menu levels).
        \item The site structure is simple, with a clear conceptual model and no unnecessary levels.
        \item The major sections of the site are available from every page (persistent navigation) and there are no dead ends.
        \item Navigation tabs are located at the top of the page, and look like clickable versions of real-world tabs.
        \item There is a site map that provides an overview of the site's content.
        \item The site map is linked to from every page.
        \item The site map provides a concise overview of the site, not a rehash of the main navigation or a list of every single topic.
        \item Good navigational feedback is provided (e.g. showing where you are in the site).
        \item Category labels accurately describe the information in the category.
        \item Links and navigation labels contain the "trigger words" that users will look for to achieve their goal.
        \item Terminology and conventions (such as link colours) are (approximately) consistent with general web usage.
        \item Links look the same in the different sections of the site.
        \item The terms used for navigation items and hypertext links are unambiguous and jargon-free.
        \item There is a visible change when the mouse points at something clickable (excluding cursor changes).
        \item Hypertext links that invoke actions (e.g downloads, new windows) are clearly distinguished from hypertext links that load another page.
        \item The site allows the user to control the pace and sequence of the interaction.
        \item There are clearly marked exits on every page allowing the user to bale out of the current task without having to go through an extended dialog.
        \item The site does not disable the browser's "Back" button and the "Back" button appears on the browser toolbar on every page.
        \item Clicking the back button always takes the user back to the page the user came from.
        \item If the site spawns new windows, these will not confuse the user (e.g. they are dialog-box sized and can be easily closed).
        \item Menu instructions, prompts and messages appear on the same place on each screen
      \end{enumerate}
  \item Forms and data entry
      \begin{enumerate}
        \item Fields in data entry screens contain default values when appropriate and show the structure of the data and the field length.
        \item Field labels on forms clearly explain what entries are desired.
        \item Text boxes on forms are the right length for the expected answer.
        \item There is a clear distinction between "required" and "optional" fields on forms.
        \item Questions on forms are grouped logically, and each group has a heading.
        \item Fields on forms contain hints, examples or model answers to demonstrate the expected input.
        \item When field labels on forms take the form of questions, the questions are stated in clear, simple language.
        \item Pull-down menus, radio buttons and check boxes are used in preference to text entry fields on forms (i.e. text entry fields are not overused).
        \item With data entry screens, the cursor is placed where the input is needed.
        \item Users can complete simple tasks by entering just essential information (with the system supplying the non-essential information by default).
        \item Forms allow users to stay with a single interaction method for as long as possible (i.e. users do not need to make numerous shifts from keyboard to mouse to keyboard)..
        \item Text entry fields indicate the amount and the format of data that needs to be entered.
        \item Forms are validated before the form is submitted .
        \item With data entry screens, the site carries out field-level checking and form-level checking at the appropriate time.
        \item The site makes it easy to correct errors (e.g. when a form is incomplete, positioning the cursor at the location where correction is required).
        \item There is consistency between data entry and data display.
        \item Labels are close to the data entry fields (e.g. labels are right justified)
      \end{enumerate}
  \item Page layout and visual design
      \begin{enumerate}
        \item The screen density is appropriate for the target users and their tasks.
        \item The layout helps focus attention on what to do next.
        \item On all pages, the most important information (such as frequently used topics, features and functions) is presented on the first screenful of information ("above the fold").
        \item The site can be used without scrolling horizontally.
        \item Things that are clickable (like buttons) are obviously pressable.
        \item Items that aren't clickable do not have characteristics that suggest that they are.
        \item The functionality of buttons and controls is obvious from their labels or from their design.
        \item Clickable images include redundant text labels (i.e. there is no 'mystery meat' navigation).
        \item Hypertext links are easy to identify (e.g. underlined) without needing to 'minesweep'.
        \item Fonts are used consistently.
        \item The relationship between controls and their actions is obvious.
        \item Icons and graphics are standard and/or intuitive (concrete and familiar).
        \item There is a clear visual "starting point" to every page.
        \item Each page on the site shares a consistent layout.
        \item Buttons and links show that they have been clicked.
        \item GUI components (like radio buttons and check boxes) are used appropriately .
        \item Fonts are readable.
        \item The site avoids italicised text and uses underlining only for hypertext links.
        \item There is a good balance between information density and use of white space.
        \item The site is pleasant to look at.
        \item Pages are free of "scroll stoppers" (headings or page elements that create the illusion that users have reached the top or bottom of a page when they have not).
        \item The site avoids extensive use of upper case text.
        \item The site has a consistent, clearly recognisable look and feel that will engage users.
        \item Saturated blue is avoided for fine detail (e.g. text, thin lines and symbols).
        \item Colour is used to structure and group items on the page.
        \item Emboldening is used to emphasise important topic categories .
        \item Pages have been designed to an underlying grid, with items and widgets aligned both horizontally and vertically.
        \item Meaningful labels, effective background colours and appropriate use of borders and white space help users identify a set of items as a discrete functional block.
        \item The colours work well together and complicated backgrounds are avoided.
        \item Individual pages are free of clutter and irrelevant information.
        \item Standard elements (such as page titles, site navigation, page navigation, privacy policy etc.) are easy to locate.
        \item The organisation's logo is placed in the same location on every page, and clicking the logo returns the user to the most logical page (e.g. the home page).
        \item Attention-attracting features (such as animation, bold colours and size differentials) are used sparingly and only where relevant.
        \item Icons are visually and conceptually distinct yet still harmonious (clearly part of the same family).
        \item Related information and functions are clustered together, and each group can be scanned in a single fixation (5-deg, about 4.4cm diameter circle on screen).
      \end{enumerate}
  \item Search Usability
      \begin{enumerate}
        \item The default search is intuitive to configure (no Boolean operators).
        \item The search results page shows the user what was searched for and it is easy to edit and resubmit the search.
        \item Search results are clear, useful and ranked by relevance.
        \item The search results page makes it clear how many results were retrieved, and the number of results per page can be configured by the user.
        \item If no results are returned, the system offers ideas or options for improving the query based on identifiable problems with the user's input.
        \item The search engine handles empty queries gracefully.
        \item The most common queries (as reflected in the site log) produce useful results.
        \item The search engine includes templates, examples or hints on how to use it effectively.
        \item The site includes a more powerful search interface available to help users refine their searches (preferably named "revise search" or "refine search", not "advanced search").
        \item The search results page does not show duplicate results (either perceived duplicates or actual duplicates).
        \item The search box is long enough to handle common query lengths.
        \item Searches cover the entire web site, not a portion of it.
        \item If the site allows users to set up a complex search, these searches can be saved and executed on a regular basis (so users can keep up-to-date with dynamic content).
        \item The search interface is located where users will expect to find it (top right of page).
        \item The search box and its controls are clearly labeled (multiple search boxes can be confusing).
        \item The site supports people who want to browse and people who want to search.
        \item The scope of the search is made explicit on the search results page and users can restrict the scope (if relevant to the task).
        \item The search results page displays useful meta-information, such as the size of the document, the date that the document was created and the file type (Word, pdf etc.).
        \item The search engine provides automatic spell checking and looks for plurals and synonyms.
        \item The search engine provides an option for similarity search ("more like this").        
      \end{enumerate}
  \item Help, feedback and error tolerance
      \begin{enumerate}
        \item The FAQ or on-line help provides step-by-step instructions to help users carry out the most important tasks.
        \item It is easy to get help in the right form and at the right time.
        \item Prompts are brief and unambiguous.
        \item The user does not need to consult user manuals or other external information to use the site.
        \item The site uses a customised 404 page, which includes tips on how to find the missing page and links to "Home" and Search.
        \item The site provides good feedback (e.g. progress indicators or messages) when needed (e.g. during checkout).
        \item User confirmation is required before carrying out potentially "dangerous" actions (e.g. deleting something).
        \item Confirmation pages are clear.
        \item Error messages contain clear instructions on what to do next.
        \item When the user needs to choose between different options (such as in a dialog box), the options are obvious.
        \item Error messages are written in a non-derisory tone and do not blame the user for the error.
        \item Pages load quickly (5 seconds or less).
        \item The site provides immediate feedback on user input or actions.
        \item Where tool tips are used, they provide useful additional help and do not simply duplicate text in the icon, link or field label.
        \item When giving instructions, pages tell users what to do rather than what to avoid doing.
        \item The site shows users how to do common tasks where appropriate (e.g. with demonstrations of the site's functionality).
        \item The site provides feedback (e.g. "Did you know?") that helps the user learn how to use the site.
        \item The site provides context sensitive help.
        \item Help is clear and direct and simply expressed in plain English, free from jargon and buzzwords.
        \item The site provides clear feedback when a task has been completed successfully.
        \item Important instructions remain on the screen while needed, and there are no hasty time outs requiring the user to write down information.
        \item Fitts' Law is followed (the distance between controls and the size of the controls is appropriate, with size proportional to distance).
        \item There is sufficient space between targets to prevent the user from hitting multiple or incorrect targets.
        \item There is a line space of at least 2 pixels between clickable items.
        \item The site makes it obvious when and where an error has occurred (e.g. when a form is incomplete, highlighting the missing fields).
        \item The site uses appropriate selection methods (e.g. pull-down menus) as an alternative to typing.
        \item The site does a good job of preventing the user from making errors.
        \item The site prompts the user before correcting erroneous input (e.g. Google's "Did you mean…?").
        \item The site ensures that work is not lost (either by the user or site error).
        \item Error messages are written in plain language with sufficient explanation of the problem.
        \item When relevant, the user can defer fixing errors until later in the task.
        \item The site can provide more detail about error messages if required.
        \item It is easy to "undo" (or "cancel") and "redo" actions.        
      \end{enumerate}
\end{itemize}



