\chapter{Conclusão}

Este trabalho apresentou uma avaliação entre quatro ferramentas CMS de código aberto desenvolvidas em Ruby on Rails. Foram escolhidos dois aspectos, usabilidade e a comunidade open source, como medidas para avaliar as ferramentas. O processo de escolha dos aspectos foi complexo, pois definir o que medir entre as ferramentas para guiar a avaliação é bastante complexo, visto que, cada CMS possui suas peculiaridades. 

As avaliações apontaram o Refinery com uma vantagem em relação às outras ferramentas, principalmente no aspecto da usabilidade. Embora tenha recebido uma avaliação melhor, a conclusão sobre uma escolha de uma ferramenta CMS deve ser levado em conta principalmente o contexto ao qual a ferramenta irá resolver o problema. Dependendo do contexto, outra ferramenta pode ser mais adequada à necessidade da aplicação. 

Durante o trabalho uma descoberta interessante foi o fato de não haver uma inspeção de checklist específica para aplicações web, inclusive é uma sugestão para trabalhos futuros, elaborar uma checklist voltada para aplicações web. Outros possíveis trabalhos futuros seriam a extensão da pesquisa para atingir mais aspectos e ferramentas, incluindo outras plataformas de desenvolvimento, trazendo aspectos como performance, cobertura de testes da ferramenta e análise da arquitetura das mesmas.  
